\chapter{Závěr}
\label{chap:zaver}


Prostudujte existující metody pro realistické zobrazování modelů vegetace v reálném čase a simulaci vybraných jevů jako je pohyb vegetace vlivem větru a změny vegetace v rámci různých ročních období. Na základě nastudovaných metod navrhněte a implementujte software umožňující realistické zobrazování vegetace v reálném čase s podporou zobrazování rozsáhlejších vegetačních celků. Výslednou implementaci otestujte z hlediska efektivity pro různé úrovně realističnosti simulace a zobrazování.

\section{Možnosti vylepšení a dalšího rozvoje}

\begin{itemize}

\item Preciznější frustum culling

\item Efektivnější řízení LOD - neurčovat LOD pro každou instanci, ale pro skupiny - využít kD-tree či BVH...

\item Optimalizace počtu vykreslovaných fragmentů LOD

\item Jiná forma LOD - billboard clouds

%%%%%%%%%%%%%%%%%%%%%%%%%%%
%	Visual quality improvements...
%
\item Plné a detailní modely (ne ty zjednodušené)

\item Order-independent průhlednost

\item Předpočítané ambient occlusion koeficienty pro listy... (listy více uvnitř koruny by měly být tmavší)

\item Barevné stínové mapy - stíny přebírají barvu listů, jimiž světlo prošlo...





\end{itemize}