%% History:
% Pavel Tvrdik (26.12.2004)
%  + initial version for PhD Report
%
% Daniel Sykora (27.01.2005)
%
% Michal Valenta (3.12.2008)
% rada zmen ve formatovani (diky M. Duškovi, J. Holubovi a J. Žďárkovi)
% sjednoceni zdrojoveho kodu pro anglickou, ceskou, bakalarskou a diplomovou praci

% One-page layout: (proof-)reading on display
%%%% \documentclass[11pt,oneside,a4paper]{book}
% Two-page layout: final printing
\documentclass[11pt,twoside,a4paper]{book}   
%=-=-=-=-=-=-=-=-=-=-=-=--=%
% The user of this template may find useful to have an alternative to these 
% officially suggested packages:
\usepackage{verbatim}
\usepackage{multirow}
\usepackage{array}
\usepackage{listings}
\usepackage{alltt}
\newcolumntype{T}{>{\vtop\bgroup\vspace*{-\ht\strutbox}\hbox\bgroup}c<{\egroup\egroup}}
\newcolumntype{M}{>{$\vcenter\bgroup\hbox\bgroup}c<{\egroup\egroup$}}

\usepackage{moreverb}
\usepackage{amsmath, amsthm, amssymb}
\usepackage[czech, english]{babel}
\usepackage[T1]{fontenc} % pouzije EC fonty 
% pripadne pisete-li cesky, pak lze zkusit take:
% \usepackage[OT1]{fontenc} 
\usepackage[utf8]{inputenc}
%=-=-=-=-=-=-=-=-=-=-=-=--=%
% In case of problems with PDF fonts, one may try to uncomment this line:
%\usepackage{lmodern}
%=-=-=-=-=-=-=-=-=-=-=-=--=%
%=-=-=-=-=-=-=-=-=-=-=-=--=%
% Depending on your particular TeX distribution and version of conversion tools 
% (dvips/dvipdf/ps2pdf), some (advanced | desperate) users may prefer to use 
% different settings.
% Please uncomment the following style and use your CSLaTeX (cslatex/pdfcslatex) 
% to process your work. Note however, this file is in UTF-8 and a conversion to 
% your native encoding may be required. Some settings below depend on babel 
% macros and should also be modified. See \selectlanguage \iflanguage.
%\usepackage{czech}  %%%%%\usepackage[T1]{czech} %%%%[IL2] [T1] [OT1]
%=-=-=-=-=-=-=-=-=-=-=-=--=%

%%%%%%%%%%%%%%%%%%%%%%%%%%%%%%%%%%%%%%%
% Styles required in your work follow %
%%%%%%%%%%%%%%%%%%%%%%%%%%%%%%%%%%%%%%%
\usepackage{graphicx}
%\usepackage{indentfirst} %1. odstavec jako v cestine.

\usepackage{k336_thesis_macros} % specialni makra pro formatovani DP a BP
 % muzete si vytvorit i sva vlastni v souboru k336_thesis_macros.sty
 % najdete  radu jednoduchych definic, ktere zde ani nejsou pouzity
 % napriklad: 
 % \newcommand{\bfig}{\begin{figure}\begin{center}}
 % \newcommand{\efig}{\end{center}\end{figure}}
 % umoznuje pouzit prikaz \bfig namisto \begin{figure}\begin{center} atd.


%%%%%%%%%%%%%%%%%%%%%%%%%%%%%%%%%%%%%
% Zvolte jednu z moznosti 
% Choose one of the following options
%%%%%%%%%%%%%%%%%%%%%%%%%%%%%%%%%%%%%
\newcommand\TypeOfWork{Diplomová práce} \typeout{Diplomova prace}
% \newcommand\TypeOfWork{Master's Thesis}   \typeout{Master's Thesis} 
% \newcommand\TypeOfWork{Bakalářská práce}  \typeout{Bakalarska prace}
% \newcommand\TypeOfWork{Bachelor's Project}  \typeout{Bachelor's Project}


%%%%%%%%%%%%%%%%%%%%%%%%%%%%%%%%%%%%%
% Zvolte jednu z moznosti 
% Choose one of the following options
%%%%%%%%%%%%%%%%%%%%%%%%%%%%%%%%%%%%%
% nabidky jsou z: http://www.fel.cvut.cz/cz/education/bk/prehled.html

%\newcommand\StudProgram{Elektrotechnika a informatika, dobíhající, Bakalářský}
%\newcommand\StudProgram{Elektrotechnika a informatika, dobíhající, Magisterský}
% \newcommand\StudProgram{Elektrotechnika a informatika, strukturovaný, Bakalářský}
 \newcommand\StudProgram{Otevřená informatika, strukturovaný, Navazující magisterský}
% \newcommand\StudProgram{Softwarové technologie a management, Bakalářský}
% English study:
% \newcommand\StudProgram{Electrical Engineering and Information Technology}  % bachelor programe
% \newcommand\StudProgram{Electrical Engineering and Information Technology}  %master program


%%%%%%%%%%%%%%%%%%%%%%%%%%%%%%%%%%%%%
% Zvolte jednu z moznosti 
% Choose one of the following options
%%%%%%%%%%%%%%%%%%%%%%%%%%%%%%%%%%%%%
% nabidky jsou z: http://www.fel.cvut.cz/cz/education/bk/prehled.html

%\newcommand\StudBranch{Výpočetní technika}   % pro program EaI bak. (dobihajici i strukt.)
\newcommand\StudBranch{Počítačová grafika a interakce}   % pro program OI mag. (strukt.)
%\newcommand\StudBranch{Softwarové inženýrství}            %pro STM
%\newcommand\StudBranch{Web a multimedia}                  % pro STM
%\newcommand\StudBranch{Computer Engineering}              % bachelor programe
%\newcommand\StudBranch{Computer Science and Engineering}  % master programe


%%%%%%%%%%%%%%%%%%%%%%%%%%%%%%%%%%%%%%%%%%%%
% Vyplnte nazev prace, autora a vedouciho
% Set up Work Title, Author and Supervisor
%%%%%%%%%%%%%%%%%%%%%%%%%%%%%%%%%%%%%%%%%%%%

\newcommand\WorkTitle{Realistické zobrazování modelů vegetace v reálném čase}
\newcommand\FirstandFamilyName{Bc. Adam Kučera}
\newcommand\Supervisor{Ing. Jiří Bittner, PhD.}


\newcommand{\lineCodeNB}[1]{\hfil\penalty 100 \hfilneg \hbox{\texttt{#1}}}
\newcommand{\lineCode}[1]{\mbox{\texttt{#1}}}

% Pouzijete-li pdflatex, tak je prijemne, kdyz bude mit vase prace
% funkcni odkazy i v pdf formatu
\usepackage[
pdftitle={\WorkTitle},
pdfauthor={\FirstandFamilyName},
bookmarks=true,
colorlinks=true,
breaklinks=true,
urlcolor=black,
citecolor= black,
linkcolor= black,
unicode=true,
]
{hyperref}



% Extension posted by Petr Dlouhy in order for better sources reference (\cite{} command) especially in Czech.
% April 2010
% See comment over \thebibliography command for details.

\usepackage[square, numbers]{natbib}             % sazba pouzite literatury
%\usepackage{url}
%\DeclareUrlCommand\url{\def\UrlLeft{<}\def\UrlRight{>}\urlstyle{tt}}  %rm/sf/tt
%\renewcommand{\emph}[1]{\textsl{#1}}    % melo by byt kurziva nebo sklonene,
\let\oldUrl\url
\renewcommand\url[1]{<\texttt{\oldUrl{#1}}>}




\begin{document}

%%%%%%%%%%%%%%%%%%%%%%%%%%%%%%%%%%%%%
% Zvolte jednu z moznosti 
% Choose one of the following options
%%%%%%%%%%%%%%%%%%%%%%%%%%%%%%%%%%%%%
\selectlanguage{czech}
%\selectlanguage{english} 

% prikaz \typeout vypise vyse uvedena nastaveni v prikazovem okne
% pro pohodlne ladeni prace


\iflanguage{czech}{
	 \typeout{************************************************}
	 \typeout{Zvoleny jazyk: cestina}
	 \typeout{Typ prace: \TypeOfWork}
	 \typeout{Studijni program: \StudProgram}
	 \typeout{Obor: \StudBranch}
	 \typeout{Jmeno: \FirstandFamilyName}
	 \typeout{Nazev prace: \WorkTitle}
	 \typeout{Vedouci prace: \Supervisor}
	 \typeout{***************************************************}
	 \newcommand\Department{Katedra počítačů}
	 \newcommand\Faculty{Fakulta elektrotechnická}
	 \newcommand\University{České vysoké učení technické v Praze}
	 \newcommand\labelSupervisor{Vedoucí práce}
	 \newcommand\labelStudProgram{Studijní program}
	 \newcommand\labelStudBranch{Obor}
}{
	 \typeout{************************************************}
	 \typeout{Language: english}
	 \typeout{Type of Work: \TypeOfWork}
	 \typeout{Study Program: \StudProgram}
	 \typeout{Study Branch: \StudBranch}
	 \typeout{Author: \FirstandFamilyName}
	 \typeout{Title: \WorkTitle}
	 \typeout{Supervisor: \Supervisor}
	 \typeout{***************************************************}
	 \newcommand\Department{Department of Computer Science and Engineering}
	 \newcommand\Faculty{Faculty of Electrical Engineering}
	 \newcommand\University{Czech Technical University in Prague}
	 \newcommand\labelSupervisor{Supervisor}
	 \newcommand\labelStudProgram{Study Programme} 
	 \newcommand\labelStudBranch{Field of Study}
}




%%%%%%%%%%%%%%%%%%%%%%%%%%    Poznamky ke kompletaci prace
% Nasledujici pasaz uzavrenou v {} ve sve praci samozrejme 
% zakomentujte nebo odstrante. 
% Ve vysledne svazane praci bude nahrazena skutecnym 
% oficialnim zadanim vasi prace.
{
\pagenumbering{roman} \cleardoublepage \thispagestyle{empty}
\chapter*{Na tomto místě bude oficiální zadání práce}
\begin{comment}
	\begin{itemize}
		\item čeština funguje? ňďťěščřžýáíéúů ŇĎŤĚŠČŘŽÝÁÍÉÚŮ,
		\item musíte si ho vyzvednout na studiijním oddělení Katedry počítačů na Karlově náměstí,
		\item v jedné odevzdané práci bude originál tohoto zadání (originál zůstává po obhajobě na katedře),
		\item ve druhé bude na stejném místě neověřená kopie tohoto dokumentu (tato se vám vrátí po obhajobě).
	\end{itemize}
\end{comment}
\newpage
}

%%%%%%%%%%%%%%%%%%%%%%%%%%    Titulni stranka / Title page 

\coverpagestarts

%%%%%%%%%%%%%%%%%%%%%%%%%%%    Podekovani / Acknowledgements 

\acknowledgements
\noindent
Velice rád bych touto cestou poděkoval vedoucímu práce Ing. Jiřímu Bittnerovi, PhD. za podnětné vedení práce.
\newline I would also like to express my thanks to Ralf Habel for inspiring consultations and providing advanced leaf texture data.



%%%%%%%%%%%%%%%%%%%%%%%%%%%   Prohlaseni / Declaration 

\declaration{V~Liberci dne \today}


%%%%%%%%%%%%%%%%%%%%%%%%%%%%    Abstract 
 
\abstractpage

Translation of Czech abstract into English.

% Prace v cestine musi krome abstraktu v anglictine obsahovat i
% abstrakt v cestine.
\vglue60mm

\noindent{\Huge \textbf{Abstrakt}}
\vskip 2.75\baselineskip

\noindent
Abstrakt v češtině

\noindent
Očekávají se cca 1 -- 2 odstavce, maximálně půl stránky.

%%%%%%%%%%%%%%%%%%%%%%%%%%%%%%%%  Obsah / Table of Contents 

\tableofcontents


%**************************************************************

\mainbodystarts
% horizontalní mezera mezi dvema odstavci
%\parskip=5pt
%11.12.2008 parskip + tolerance
\normalfont
\parskip=0.2\baselineskip plus 0.2\baselineskip minus 0.1\baselineskip

% Odsazeni prvniho radku odstavce resi class book (neaplikuje se na prvni 
% odstavce kapitol, sekci, podsekci atd.) Viz usepackage{indentfirst}.
% Chcete-li selektivne zamezit odsazeni 1. radku nektereho odstavce,
% pouzijte prikaz \noindent.

%**************************************************************

% Pro snadnejsi praci s vetsimi texty je rozumne tyto rozdelit
% do samostatnych souboru nejlepe dle kapitol a tyto potom vkladat
% pomoci prikazu \include{jmeno_souboru.tex} nebo \include{jmeno_souboru}.
% Napr.:
% \include{1_uvod}
% \include{2_teorie}
% atd...

%*ÚVOD****************************************************************************
%\include{chapters/01_uvod}


%*Popis problému, specifikace cíle****************************************************************************
%\include{chapters/02_problem_cile}

%*Analýza a návrh řešení****************************************************************************
%\chapter{Analýza a návrh řešení}
\label{chap:analyza}
%Analýza a návrh implementace (včetně diskuse různých alternativ a volby implementačního prostředí).


%*****************************************************************************
\chapter{Realizace}
\label{chap:realizace}

%%%%%%%%%%%%%%%%%%%%%%%%%%%%%%%%%%%%%%%%%
\section{Analýza modelu a předgenerování dat}
\label{sec-modelAnalysis}
\begin{itemize}
 \item jak z modelu získat potřebná data pro animaci - rekonstrukce topologie stromu
 \item generování dat potřebných pro nižší LOD
 \item offscreen rendering z různýh směrů a v různých intervalech near-far plane - orthogonální projekce
\end{itemize}

%%%%%%%%%%%%%%%%%%%%%%%%%%%%%%%%%%%%%%%%%
\section{Architektura LOD}
\label{sec-LODarchitecture}

\begin{itemize}
 \item skupiny instancí - instance v přechodu ošetřovat individuálně (není jich tolik), instance bez přechodu  vykreslovat po skupinách - 
\item renderqueues
\item instancování
\item průhlednost
\end{itemize}

%%%%%%%%%%%%%%%%%%%%%%%%%%%%%%%%%%%%%%%%%
\section{Řízení LOD}
\label{sec-LODcontrol}

\begin{itemize}
\item skrývání - řízení průhlednosti
\item pořadí vykreslování primitiv \& instancování
\end{itemize}

%%%%%%%%%%%%%%%%%%%%%%%%%%%%%%%%%%%%%%%%%
\section{Zobrazení geometrického modelu}
\label{sec-3Ddisplay}

\begin{itemize}
\item animace
\item listy
\end{itemize}

%%%%%%%%%%%%%%%%%%%%%%%%%%%%%%%%%%%%%%%%%
\section{Zobrazení nižších LOD}
\label{sec-LODdisplay}

\begin{itemize}
\item skrývání - řízení průhlednosti
\item pořadí vykreslování primitiv \& instancování
\end{itemize}


%*****************************************************************************
%\chapter{Testování}
\label{chap:testovani}

%%%%%%%%%%%%%%%%%%%%%%%%%%%%%%%%%%%%%
%	LOD0 vs LOD1
%
obrázek scény A
\begin{figure}[here]
\begin{center}
$\begin{array}{ccc}
\includegraphics[width=0.3\textwidth]{./testing/LOD0only50.png}&
\includegraphics[width=0.3\textwidth]{./testing/LOD1only50.png}&
\includegraphics[width=0.3\textwidth]{./testing/LOD2only50.png}
\\
(a)&(b)&(c)
\end{array}$
\end{center}
\caption[Náhledy testovací scény]%
{Náhledy z testování, scéna SMALL FOREST s 50 instancemi, (a) LOD0,  (b) LOD1, (c) LOD2, 0\label{fig:testONLYsmal}
}
\end{figure}
obrázek scény B
\begin{figure}[here]
\begin{center}
$\begin{array}{ccc}
\includegraphics[width=0.3\textwidth]{./testing/LOD0only1000.png}&
\includegraphics[width=0.3\textwidth]{./testing/LOD1only1000.png}&
\includegraphics[width=0.3\textwidth]{./testing/LOD2only1000.png}
\\
(a)&(b)&(c)
\end{array}$
\end{center}
\caption[Náhledy testovací scény]%
{Náhledy z testování, scéna FOREST s 1000 instancemi , (a) LOD0,  (b) LOD1, (c) LOD2, 0\label{fig:testONLYl}
}
\end{figure}

porovnání kvality při pohledu z dálky

\begin{figure}[here]
\begin{tabular}{r l}
LOD0 & \includegraphics[width=0.9\textwidth]{./testing/LOD0only1000.png}\\
LOD1 & \includegraphics[width=0.9\textwidth]{./testing/LOD1only1000.png}\\
\end{tabular}
\caption[Náhledy testovací scény]%
{Náhledy z testování, scéna FOREST s 1000 instancemi, porovnání výsledné kvality \label{fig:testQuality}
}
\end{figure}



\begin{table}[here]
\centering
\begin{tabular}{|r | c | c | l |} 
\hline 
\#instancí & LOD0 (ms)& LOD1 (ms)& \\ [0.5ex] 
\hline
10	&	2,98	&	3,71	& 	\multirow{3}{*}{scéna A}\\
25	&	7,37		&	8,96	&	 \\
50	&	13,71	&	16,89	&	 \\
\hline
100	& 	20,25	&	6,54	&	\multirow{4}{*}{scéna B}\\
250	& 	48,15	&	15,59	&\\
500	& 	94,73 	& 	30,36	&\\
1000 & 	188,96 	& 	57,16	&\\
 [1ex] 
\hline 
\end{tabular}
\label{table:lod01-1MS}
\caption{Porovnání LOD0 a LOD1, bez multisamplingu}

\end{table}

\begin{table}[here]
\centering
\begin{tabular}{|r | c | c | l |} 
\hline 
\#instancí & LOD0 (ms)& LOD1 (ms)& \\ [0.5ex] 
\hline
10	&	3,72	&	4,26	& 	\multirow{3}{*}{scéna A}\\
25	&	8,32	&	9,48	&	 \\
50	&	15,94	&	16,5	&	 \\
\hline
100	& 	19,34	&	6,04	&	\multirow{4}{*}{scéna B}\\
250	& 	47,5		&	14,36	&\\
500	& 	94,2 	& 	28,58	&\\
1000 & 	188,3 	& 	55,87	&\\
 [1ex] 
\hline 
\end{tabular}
\label{table:lod01-4MS}
\caption{Porovnání LOD0 a LOD1, 4x multisampling}
\end{table}


%%%%%%%%%%%%%%%%%%%%%%%%%%%%%%%%%%%%%
%	Fragmenty vs čas, 1 strom různé vzdálenosti
%
\begin{table}[here]
\centering
\begin{tabular}{|r | c | c | c | c | c | c |} 

\hline 
& \multicolumn{3}{|c|}{LOD0}& \multicolumn{3}{|c|}{LOD1}\\
vzdálenost & \# fragmentů & fps & čas (ms) & \# fragmentů & fps & čas (ms)\\ [0.5ex] 
\hline
50		&6734		&1693,29	&0,59	&4226		&1391,3		&0,71875\\
40		&10537		&1538,84	&0,65	&7662		&1268,64	&0,78825\\
30		&18886		&1263,75	&0,79	&15379		&1183,22	&0,84515\\
20		&43139		&928,04		&1,08	&39297		&1018,22	&0,98211\\
10		&180614	&389,14		&2,57	&185531	&665,14		&1,50345\\
 [1ex] 
\hline 
\end{tabular}
\label{table:lod01-fragtime}
\caption{Závislost zobrazovacího času na počtu fragmentů}

\end{table}
nahledy
\begin{figure}[here]
\begin{center}
$\begin{array}{ccccc}
\includegraphics[width=0.17\textwidth]{./testing/LOD1-d50.png}&
\includegraphics[width=0.17\textwidth]{./testing/LOD1-d40.png}&
\includegraphics[width=0.17\textwidth]{./testing/LOD1-d30.png}&
\includegraphics[width=0.17\textwidth]{./testing/LOD1-d20.png}&
\includegraphics[width=0.17\textwidth]{./testing/LOD1-d10.png}
\\
(a)&(b)&(c)&(d)&(e)
\end{array}$
\end{center}
\caption[Náhledy testovací scény]%
{Náhledy z testování, scéna FRAGMENTS, (a) vzdálenost 50,  (b) vzdálenost 40, (c) vzdálenost 30, (d) vzdálenost 20, (e) vzdálenost 10\label{fig:testFRAG}
}
\end{figure}

grafy

\begin{figure}[here]
\begin{center}
$\begin{array}{cc}
\includegraphics[width=0.5\textwidth]{./graphs/fragLOD0.png}&
\includegraphics[width=0.5\textwidth]{./graphs/fragLOD1.png}
\\
(a)&(b)
\end{array}$
\end{center}
\caption[Grafy závislosti zobrazovacího času na počtu fragmentů]%
{Grafy závislosti zobrazovacího času na počtu fragmentů\label{fig:testFRAG}
}
\end{figure}



%%%%%%%%%%%%%%%%%%%%%%%%%%%%%%%%%%%%%
%	Podil ruznych LOD na vyslednem case
%
\begin{table}[here]
\centering
\begin{tabular}{|r | c | c | c | c |} 
\hline 
 & fps & čas (ms) & rozdíl časů & \% \\
\hline
bez stromů		&2384,76	&0,42	&0,42	&1,46\\
LOD0			&112,64		&8,88	&8,46	&29,43\\
+LOD1			&36,7		&27,25	&18,37	&63,92\\
+LOD2			&34,79		&28,74	&1,49	&5,19\\
[1ex] 
\hline 
\end{tabular}
\label{table:lod012-contribs}
\caption{Podíly jednotlivých LOD na výsledném čase zobrazení}
\end{table}

\begin{figure}[here]
\begin{center}
\includegraphics[width=0.6\textwidth]{./graphs/LODsContrib.png}
\end{center}
\caption[Graf podílů jednotlivých LOD na zobrazovacím čase]%
{Graf podílů jednotlivých LOD na zobrazovacím čase\label{fig:testCONTR}
}
\end{figure}

\begin{figure}[here]
\begin{center}
$\begin{array}{ccc}
\includegraphics[width=0.31\textwidth]{./testing/on-offLOD0.png}&
\includegraphics[width=0.31\textwidth]{./testing/on-offLOD1.png}&
\includegraphics[width=0.31\textwidth]{./testing/on-offLOD2.png}
\\
(a)&(b)&(c)
\end{array}$
\end{center}
\caption[Náhledy testovací scény]%
{Náhledy testovací scény. Zobrazení pouze LOD0 (a), LOD0 a LOD1 (b), všechny LOD (c)\label{fig:testFRAG}
}
\end{figure}

%%%%%%%%%%%%%%%%%%%%%%%%%%%%%%%%%%%%%
%	Ruzne urovne animace
%
\begin{table}[here]
\centering
\begin{tabular}{|r | c | c | c | c || c | c | c | c |} 
\hline 
		&\multicolumn{4}{|c||}{LOD1}		&\multicolumn{4}{c|}{LOD2}\\	
		&\multicolumn{4}{|c||}{\# instancí}	&\multicolumn{4}{c|}{\# instancí}\\
			&10		&25		&50		&100	&10		&25		&50		&100\\
\hline					
bez animace	&1,55	&3,39	&6,35	&11,31	&0,68	&1		&1,73	&2,68 \\
kmen		&2,64	&4,51	&8,14	&13,96	&1,35	&1,74	&3,07	&4,47\\
hlavní větve	&3,84	&7,73	&14,22	&23,53	&1,39	&3,06	&3,83	&4,75\\
listy			&4		&8,45	&16,91	&30,93	&1,79	&3,66	&4,14	&6,44\\
[1ex] 
\hline 
\end{tabular}
\label{table:lod12-anim}
\caption{Vliv podrobnosti animace na zobrazovací čas}
\end{table}

\begin{figure}[here]
\begin{center}
\includegraphics[width=0.75\textwidth]{./graphs/animLOD1.png}
\end{center}
\caption[Graf závislosti zobrazovacího času na podrobnosti animace LOD1]%
{Graf závislosti zobrazovacího času na podrobnosti animace LOD1.\label{fig:testANIM1}
}
\end{figure}
\begin{figure}[here]
\begin{center}
\includegraphics[width=0.75\textwidth]{./graphs/animLOD2.png}
\end{center}
\caption[Graf závislosti zobrazovacího času na podrobnosti animace LOD2]%
{Graf závislosti zobrazovacího času na podrobnosti animace LOD2.\label{fig:testANIM2}
}
\end{figure}


%%%%%%%%%%%%%%%%%%%%%%%%%%%%%%%%%%%%%
%	Shadow mapping
%
\begin{table}[here]
\centering
\begin{tabular}{|r | c | c | c |} 
\hline 
&\multicolumn{3}{|c|}{zobrazovací čas (ms)}\\
rozlišení stínové mapy			&LOD0		&LOD1 (přepis hloubky)	&LOD1 (bez přepisu)\\
\hline					
512 $\times$ 512	&2,26	&3,21	&3,28 \\
1024 $\times$ 1024	&2,44	&3,08	&3,02\\ 
2048 $\times$ 2048	&2,75	&5,98	&5,74\\
[1ex] 
\hline 
\end{tabular}
\label{table:lod01-shadow}
\caption{Vliv rozlišení stínové mapy na zobrazovací čas}
\end{table}

\begin{itemize}
 \item Způsob, průběh a výsledky testování.
 \item Srovnání s existujícími řešeními, pokud jsou známy.
\end{itemize} 


%*****************************************************************************
\chapter{Závěr}
\label{chap:zaver}


Prostudujte existující metody pro realistické zobrazování modelů vegetace v reálném čase a simulaci vybraných jevů jako je pohyb vegetace vlivem větru a změny vegetace v rámci různých ročních období. Na základě nastudovaných metod navrhněte a implementujte software umožňující realistické zobrazování vegetace v reálném čase s podporou zobrazování rozsáhlejších vegetačních celků. Výslednou implementaci otestujte z hlediska efektivity pro různé úrovně realističnosti simulace a zobrazování.

\section{Možnosti vylepšení a dalšího rozvoje}

\begin{itemize}

\item Preciznější frustum culling

\item Efektivnější řízení LOD - neurčovat LOD pro každou instanci, ale pro skupiny - využít kD-tree či BVH...

\item Optimalizace počtu vykreslovaných fragmentů LOD

\item Jiná forma LOD - billboard clouds

%%%%%%%%%%%%%%%%%%%%%%%%%%%
%	Visual quality improvements...
%
\item Plné a detailní modely (ne ty zjednodušené)

\item Order-independent průhlednost

\item Předpočítané ambient occlusion koeficienty pro listy... (listy více uvnitř koruny by měly být tmavší)

\item Barevné stínové mapy - stíny přebírají barvu listů, jimiž světlo prošlo...





\end{itemize}

\begin{itemize}
\item Zhodnocení splnění cílů DP/BP a  vlastního přínosu práce (při formulaci je třeba vzít v potaz zadání práce).
\item Diskuse dalšího možného pokračování práce.
\end{itemize} 



%*****************************************************************************
% Seznam literatury je v samostatnem souboru reference.bib. Ten
% upravte dle vlastnich potreb, potom zpracujte (a do textu
% zapracujte) pomoci prikazu bibtex a nasledne pdflatex (nebo
% latex). Druhy z nich alespon 2x, aby se poresily odkazy.

% originally following specification for bibliography formating was used
%\bibliographystyle{abbrv}

% Here is an improvment by Petr Dlouhy (April 2010).
% It is mainly for supervisors who expect Czech fomrating rules for references
% Additional feature is live url addresses to sources from your pdf file
% It requires the file csplainnat.bst (included in this sample zipfile).

\bibliographystyle{csplainnat}

%bibliographystyle{plain}
%\bibliographystyle{psc}
{
%JZ: 11.12.2008 Kdo chce mit v techto ukazkovych odkazech take odkaz na CSTeX:
\def\CS{$\cal C\kern-0.1667em\lower.5ex\hbox{$\cal S$}\kern-0.075em $}
\bibliography{reference}
 }

% M. Dušek radi:
%\bibliographystyle{alpha}
% kdy citace ma tvar [AutorRok] (napriklad [Cook97]). Sice to asi neni  podle ceske normy (BTW BibTeX stejne neodpovida ceske norme), ale je to nejprehlednejsi.
% 3.5.2009 JZ polemizuje: BibTeX neobvinujte, napiste a poskytnete nam styl (.bst) splnujici citacni normu CSN/ISO.

%*****************************************************************************
%%%%%%%%%%%%%%%%%%%%%%%%%%%%%%%  Seznam obrazku / List of Figures 

\listoffigures


%%%%%%%%%%%%%%%%%%%%%%%%%%%%%%%  Seznam tabulek / List of Tables

\listoftables



%*****************************************************************************
\appendix

%*****************************************************************************
\chapter{Seznam použitých zkratek}

\begin{description}
\item[2D] 		Two-Dimensional
\item[3D] 		Three-Dimensional
\item[BRDF] 	Bidirectional Reflectance Distribution Function
\item[BSSRDF] 	Bidirectional Surface Scattering Reflectance Distribution Function
\item[CPU] 		Central Processing Unit
\item[fps] 		Frames Per Second
\item[GPU] 		Graphics Processing Unit
\item[LOD]		Level Of Detail
\item[OpenGL] 	Open Graphic Library

\end{description}

%*****************************************************************************
%\chapter{UML diagramy}
%\textbf{\large Tato příloha není povinná a zřejmě se neobjeví v každé práci. Máte-li ale větší množství podobných diagramů popisujících systém, není nutné všechny umísťovat do hlavního textu, zvláště pokud by to snižovalo jeho čitelnost.}

%*****************************************************************************
\chapter{Instalační a uživatelská příručka}
\textbf{\large Tato příloha velmi žádoucí zejména u softwarových implementačních prací.}

%*****************************************************************************
\chapter{Obsah přiloženého CD}
\textbf{\large Tato příloha je povinná pro každou práci. Každá práce musí totiž obsahovat přiložené CD. Viz dále.}

%\begin{figure}[h]
%\begin{center}
%\includegraphics[width=14cm]{figures/seznamcd}
%\caption{Seznam přiloženého CD --- příklad}
%\label{fig:seznamcd}
%\end{center}
%\end{figure}

\end{document}
